%% ---------------------------------------------------------------------------
%% paNotation.tex
%%
%% Notation
%%
%% $Id: paNotation.tex,v 1.15 2004/03/30 05:55:59 alvarado Exp $
%% ---------------------------------------------------------------------------

\cleardoublepage
\renewcommand{\nomname}{List of symbols and abbreviations}
\markboth{\nomname}{\nomname}
\renewcommand{\nompreamble}{\addcontentsline{toc}{chapter}{\nomname}%
\setlength{\nomitemsep}{-\parsep}
\setlength{\itemsep}{10ex}
}

%%
% Símbolos en la notación general
% (es posible poner la declaración en el texto
%%

\symg[t]{$\sys{\cdot}$}{Transformation performed by a system}
\symg[yscalar]{$y$}{Scalar.}
\symg[zconjugado]{$\conj{z}$}{Conjugate Complex of $z$}
\symg[rcomplexreal]{$\Re(z)$ o $z_{\Re}$}{Real part of the complex number $z$}
\symg[icompleximag]{$\Im(z)$ o $z_{\Im}$}{Imaginary part of the complex number $z$}
\symg[jimaginario]{$j$}{$j=\sqrt{-1}$}
\symg[xvector]{$\vct{x}$}{Vector. \newline\hspace{1mm}%
  $\vct{x}=\left[ x_1 \; x_2 \; \ldots \; x_n \right]^T =
  \begin{bmatrix}
    x_1 \\ x_2 \\ \vdots \\ x_n
  \end{bmatrix}$}

\symg[amatrix]{$\mat{A}$}{Matrix. \newline\hspace{1mm}%
  $\mat{A} =
  \begin{bmatrix}
    a_{11} & a_{12} & \cdots & a_{1m}\\
    a_{21} & a_{22} & \cdots & a_{2m}\\
    \vdots & \vdots & \ddots & \vdots\\
    a_{n1} & a_{n2} & \cdots & a_{nm}\\
  \end{bmatrix}$}

\symg[C]{$\setC$}{Complex numbers set.}

%%
% Algunas abreviaciones
%%

\syma{MLP}{Multilayer perceptrons}
\syma{CNN}{Convolutional Neural Networks}
\syma{VAE}{Variational Autoencoder}
\syma{sVAE}{Spatial Variational Autoencoder}
\syma{GM-VAE}{Gaussian-Mixture Variational Autoencoder}
\syma{GAN}{Generative Adversarial Networks}
\syma{KL}{Kullback-Leibler}
\syma{MVN}{Matrix-variable normal distribution}

\printnomenclature[20mm]

%%% Local Variables:
%%% mode: latex
%%% TeX-master: "paMain"
%%% End:
