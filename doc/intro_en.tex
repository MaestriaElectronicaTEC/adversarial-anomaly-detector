%% ---------------------------------------------------------------------------
%% intro.tex
%%
%% Introduction
%%
%% $Id: intro.tex 1477 2019-11-28 15:10:43Z lumurillo $
%% ---------------------------------------------------------------------------

\chapter{Introduction}
\label{chp:intro}

The diseases and pests control are becoming more relevant in the latest years due to the consequences that climate change brings, sucha as \cite{Coakley1999} the alteration of stages and rates of development of pathogens, the modification of host resistance and changes in the physiology of host-pathogen interactions. These issues cause changes in the geographical distribution and growth of plant species.

All these situations have the potential to cause catastrophic plant diseases, leading to the loss of food crops, which according to \cite{Strange2005} aggravates the deficit of food supply. In this sense, the tomato has a key role, due to its impact in the diet of people. As stated in \cite{Bergougnoux2014}, the tomato represents the most economically important vegetable crop worldwide due to the precense in the diet of millions of people, hence being one of the most produced vegetables after potatoes and before onions.

In \cite{Park2017} is mentioned the use of non-invasive methods to deal with crop disease. Normally, to combat this problem, the farmers use chemical agents as pesticides which has environmental consequences and health issues to the people that consume the vegetable.

In Costa Rica, the disease that causes the most damage to the tomato crops is the \emph{Phytophthora infestans} and the most important pest in the whiteflies \cite{MAG2007}.

In the development of automated applications for disease detection requires the use of a labeled dataset, which implies invest time consulting with experts for the labeling process. For that reason, this work explores methods to reduce this time as much as possible. The approach followed is the detection of anomalies where a set of data, in its majority, represents the healthy plants and the outliers can be seen as the affected plants or the anomalies.

The general goal of this project is the evaluation of different algorithms for anomaly detection, in the context of tomato images. The specific goals are the following:

\begin{itemize}
 \item To design a dataset to train the machine learning models.
 \item Selection of an anomaly detection algorithms.
 \item To evaluate the selected methods for detection anomaly in tomato images.
\end{itemize}

The rest of the work is structured as follows:

The state of the art chapter \ref{ch:marco} covers related works for the detection of anomalies in different contexts like in agriculture or health care; as well as additional concepts that underline this project.

In chapter \ref{ch:solution} the solution strategy to evaluate different deep learning architectures able to detect anomalies is described.

Chapter \ref{ch:results} presents a discussion of the performance of the different architecture and which of them are the best option for the detection of diseases in tomato.

Finally, the conclusions and the future work are summarized in chapter \ref{ch:conclusions}.

%%% Local Variables: 
%%% mode: latex
%%% TeX-master: "main"
%%% End: 
