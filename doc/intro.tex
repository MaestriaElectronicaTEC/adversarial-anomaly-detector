%% ---------------------------------------------------------------------------
%% intro.tex
%%
%% Introduction
%%
%% $Id: intro.tex 1477 2010-07-28 21:34:43Z palvarado $
%% ---------------------------------------------------------------------------

\chapter{Introducción}
\label{chp:intro}

En la \nt{introducción} deben quedar completamente claros los siguientes
aspectos, cuyo significado depende del tipo concreto de tesis:

\begin{compactitem}
\item Contexto
\item Problema
\item Esbozo de solución
\item Objetivos y estructura
\end{compactitem}

El contexto corresponde al entorno donde se desarrolla el proyecto de
tesis, que puede ser el área general de aplicación, un dominio de
problemas, etc. El problema concreto se sintetiza usualmente en una
frase o pregunta. Esta pregunta debería ser una consecuencia a la que
se llega después de realizar el desarrollo del contexto. Del
planteamiento del problema se deriva cuál es el objetivo del trabajo
en particular, que a su vez debe conducir al lector de forma natural
al esbozo de la solución del problema a tratar en la
tesis. Generalmente para aclarar la solución se hace uso de un
diagrama de bloques o diagrama de flujo general, es decir, desde un
nivel de abstracción alto, donde no sea necesario entrar en detalles
técnicos. Usualmente este diagrama y su breve explicación dictan cuál
debe ser la estructura del resto del tesis, que es mencionada siempre
al final de la introducción.

Una buena introducción debe lograr que el lector tenga interés de leer el resto
del tesis.

Es recomendable dividir la tesis en secciones, nombradas cada una de acuerdo a
su contenido. \textbf{Jamás} utilice los nombres de la guía como
``\emph{Problema existente e importancia de su solución}'', sino algo como ``La
deforestación en Costa Rica'' o lo que se adecúe a su problema en particular.

Recuerde que en español solo la primera letra del título va en mayúscula
(exceptuando nombres propios, por supuesto).

\section{Objetivos y estructura del documento}

\index{objetivos}
Esta plantilla LaTeX tiene como objetivo simplificar la construcción del
documento de tesis, presentando ejemplo de figuras y tablas, así como otorgar
una plataforma de compilación en GNU/Linux que simplifique la administración de
todo el documento.

La última sección de la introducción usualmente sí tiene un título estandar que
es ``Objetivos y estructura del documento'', donde se presentan \emph{en prosa}
los objetivos general y específicos que ha tenido el proyecto de tesis,
así como la estructura de la tesis (por ejemplo, ``en el siguiente capítulo se
esbozan los fundamentos teóricos necesarios para explicar en el
capítulo~\ref{ch:solucion} la propuesta realizada$\ldots$''

%%% Local Variables: 
%%% mode: latex
%%% TeX-master: "main"
%%% End: 
