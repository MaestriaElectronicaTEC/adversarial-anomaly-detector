\chapter*{Resumen}
\thispagestyle{empty}

El tomate es una de las principales vegetales a nivel mundial debido a su versatilidad de uso y a su impacto economico. Sin embargo el cambio climático ha provocado que el manejo de plagas y enfermedades sea cada vez más complicada. Es por ello que es importante la implementación de técnicas no invasivas para el diagnóstico temprano de enfermedades en el campo de cultivo. En este proyecto se presenta un estudio de algorítmos no supervisados tales como los modelos generativos, con el objetivo de detectar anomalías en fotos de tomate. Además se plantea una propuesta de modelo capaz de detectar anomalias, basandose en los redes generativas adversarias.

\bigskip

\textbf{Palabras clave:} \thesisKeywords

\clearpage
\chapter*{Abstract}
\thispagestyle{empty}

Tomato is one of the main vegetables worldwide due to its versatility of use and its economic impact. However, climate change has caused the management of pests and diseases to be increasingly complicated. That is why it is important to implement non-invasive techniques for the early diagnosis of diseases in the crop field. This project presents a study of unsupervised algorithms such as generative models, with the aim of detecting anomalies in tomato photos. In addition, a model proposal capable of detecting anomalies is presented, based on the adversary generative networks.

\bigskip

\textbf{Keywords:} \thesisKeywords 

\cleardoublepage

%%% Local Variables: 
%%% mode: latex
%%% TeX-master: "main"
%%% End: 
