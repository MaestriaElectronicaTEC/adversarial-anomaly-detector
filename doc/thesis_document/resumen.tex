\chapter*{Resumen}
\thispagestyle{empty}

El tomate es uno de los principales vegetales a nivel mundial debido a su versatilidad de uso y a su impacto económico. Sin embargo, el cambio climático ha provocado que el manejo de plagas y enfermedades sea cada vez más complicado. Es por ello que la implementación de técnicas no invasivas para el diagnóstico temprano de enfermedades en el campo de cultivo representa una solución viable para el control de plagas y enfermedades, evitando efectos secundarios tales como afecciones al medio ambiente. En este proyecto se presenta un estudio de algorítmos semi-supervisados tales como los modelos generativos, con el objetivo de detectar anomalías en grafías de tomate. Además se plantea una propuesta de modelo capaz de detectar anomalias, basandose en las redes generativas adversarias.

\bigskip

\textbf{Palabras clave:} Tomate, aprendizaje profundo, detección de anomalías, auto-codificadores, redes generativas adversarias

\clearpage
\chapter*{Abstract}
\thispagestyle{empty}

Tomato is one of the main vegetables worldwide due to its versatility of use and its economic impact. However, climate change has caused the management of pests and diseases to be increasingly complicated. That is why the implementation of non-invasive techniques for the early diagnosis of diseases in the field of crops represents a viable solution for the control of pests and diseases, avoiding side effects such as environmental conditions. This project presents a study of semi-supervised algorithms such as generative models, with the aim of detecting anomalies in tomato spelling. In addition, a model proposal capable of detecting anomalies is proposed, based on adversary generative networks.

\bigskip

\textbf{Keywords:} \thesisKeywords 

\cleardoublepage

%%% Local Variables: 
%%% mode: latex
%%% TeX-master: "main"
%%% End: 
