%%%%%%%%%%%%%%%%%%%%%%%%%%%%%%%%%%%%%%%%%%%%%%%%%%%%%%%%%%%%%%%%%%%%%%%%%%%%%%%
% Author:  Pablo Alvarado
%
% Escuela de Electrónica
% Instituto Tecnológico de Costa Rica
%
% Phone:   +506 550 2106
% Fax:     +506 591 6629
% email:   palvarado@ietec.org
%
% $Id: macros.tex 1497 2010-08-09 17:04:26Z palvarado $
%
%%%%%%%%%%%%%%%%%%%%%%%%%%%%%%%%%%%%%%%%%%%%%%%%%%%%%%%%%%%%%%%%%%%%%%%%%%%%%%%

% Configuration of the exercises package, which is used to collect all
% problems and answers in the document.
\usepackage[exercisedelayed,answerdelayed,lastexercise]{exercise}
\renewcounter{Exercise}[chapter]
\renewcommand{\ExerciseName}{Problema}
\renewcommand{\theExercise}{\thechapter.\arabic{Exercise}}
\newcommand{\ExerciseLabel}{Exercise.\theExercise}
\renewcommand{\ExerciseHeader}%
{\textbf{\ExerciseName\ \theExercise.\ \ExerciseHeaderTitle\ }}
\renewcommand{\AnswerHeader}%
{\textbf{\ExerciseName\ \theExercise.\ }}


\usepackage{ifpdf}

% Command to change between draft or release mode:
\newcommand{\ifdraft}[2]{\ifthenelse{\boolean{draftmode}}{#1}{#2}}
% Command to change between draft or release mode:
\newcommand{\ifbook}[2]{\ifthenelse{\boolean{bookmode}}{#1}{#2}}

% include all required packages here
\usepackage{csquotes}                     % recommended for biblatex
\usepackage[spanish]{babel}               % supports english, but default is 
                                          % spanish...
% \newcommand*{\SelectSpanish}{%          % well, the last line indeed selects
%   \hyphenrules{spanish}%                % english over spanish, but with this
%   \languageshorthands{spanish}%         % command we turn it around.
%   \captionsspanish                      % The reason: hyperref has some
%   \datespanish                          % problems with the spanish babel,
% }                                       % so we use some trick here so that it
% \AtBeginDocument{\SelectSpanish}        % thinks it is english.

\usepackage{makeidx}                    % to create index file

\ifdraft{%
  %\usepackage[refpage]{nomencl}        % Use to easily administrate the list
 \usepackage{nomencl}                   % of symbols
}{%
 \usepackage{nomencl}
}
%\usepackage{times}                     % replace latex pk fonts with ps type I
                                        % don't forget to use dvips -D600 -Pcmz
                                        % to ensure Type I fonts!
\usepackage{amsmath}
\usepackage{amssymb,amstext}            % AMS-math and symbols package
\usepackage{mathrsfs}                   % Calygraphic fonts for transforms
\usepackage{array}                      % extensions to tabular environment
\usepackage{longtable}                  % supports extraordinary long tables
\usepackage{tabularx}                   % supports tables with fixed width
\usepackage{afterpage}                  % put something only after the page
\usepackage{multirow}                   % supports multiple row grouping in 
                                        % tables
\usepackage{multicol}                   % multiple columns environments
%\usepackage{paralist}                  % a few enumeration settings (old)
\usepackage{enumitem}                   % better enumeration with paralist 
                                        % equivalents as follows:

\newlist{compactitem}{itemize}{3}
\setlist[compactitem]{topsep=0pt,partopsep=0pt,itemsep=0pt,parsep=0pt}
\setlist[compactitem,1]{label=\textbullet}
\setlist[compactitem,2]{label=---}
\setlist[compactitem,3]{label=*}

\newlist{compactdesc}{description}{3}
\setlist[compactdesc]{topsep=0pt,partopsep=0pt,itemsep=0pt,parsep=0pt}

\newlist{compactenum}{enumerate}{3}
\setlist[compactenum]{topsep=0pt,partopsep=0pt,itemsep=0pt,parsep=0pt}

\usepackage{icomma}                     % decimal comma in math mode

\usepackage{bold-extra}

\usepackage[format=hang,%
            font=small,%
            labelfont=bf]{caption}      % nicer figure captions
%\usepackage{sty/ftcap}                 % switch \abovecaptionskip and
%                                       % \belowcaptionskip for tables, in 
%                                       % order to avoid the caption to be
%                                       % too near to the table itself
% locally added packages
\usepackage{float}                      % really place figures "here" (H)
\usepackage{booktabs}                   % book type tabulars

% the own style with options depending on the draft mode
\ifdraft{%
\usepackage[todo]{sty/tecStyle}         % some command definitions
                                        % options [todo] todo-index
}{%
\usepackage{sty/tecStyle}               % some command definitions
                                        % options [todo] todo-index
}


%% fix the title for examples
\renewcommand{\examplelistname}{Índice de ejemplos}
\renewcommand{\examplename}{Ejemplo}

%% define some command to cope with the tribunal names

%% Lector I

\newcommand{\nameLectorI}{$<$\emph{Use setLectorI in main.tex}$>$}
\newcommand{\genderLectorI}{$<$\emph{Use setLectorI in main.tex}$>$}

\newcommand{\setLectorI}[1][M]{%
  \ifthenelse{\equal{#1}{F}}{%
    \renewcommand{\genderLectorI}{Profesora Lectora}%
  }{%
    \renewcommand{\genderLectorI}{Profesor Lector}
  }
  \lectorIRelay
}

\newcommand{\lectorIRelay}[1]{%
  \renewcommand{\nameLectorI}{#1}
}

%% Lector II

\newcommand{\nameLectorII}{$<$\emph{Use setLectorII in main.tex}$>$}
\newcommand{\genderLectorII}{$<$\emph{Use setLectorII in main.tex}$>$}

\newcommand{\setLectorII}[1][M]{%
  \ifthenelse{\equal{#1}{F}}{%
    \renewcommand{\genderLectorII}{Profesora Lectora}%
  }{%
    \renewcommand{\genderLectorII}{Profesor Lector}
  }
  \lectorIIRelay
}

\newcommand{\lectorIIRelay}[1]{%
  \renewcommand{\nameLectorII}{#1}
}

%% Asesor

\newcommand{\nameAsesor}{$<$\emph{Use setAsesor in main.tex}$>$}
\newcommand{\genderAsesor}{$<$\emph{Use setAsesor in main.tex}$>$}

\newcommand{\setAsesor}[1][M]{%
  \ifthenelse{\equal{#1}{F}}{%
    \renewcommand{\genderAsesor}{Profesora Asesora}%
  }{%
    \renewcommand{\genderAsesor}{Profesor Asesor}
  }
  \asesorRelay
}

\newcommand{\asesorRelay}[1]{%
  \renewcommand{\nameAsesor}{#1}
}





\usepackage{url}                        % allows linebreaks at certain
                                        % characters or combinations of 
                                        % characters for URLs

\usepackage[nottoc]{tocbibind}          % Fix the hyperrefs to TOC,TOF, etc.
                                        % and ensure that they appear all in 
                                        % the Table of Contents

% For pdflatex
% - The hyperref package should always be loaded last, since it has to
%   overwrite some of the commands.
% - The package subfigure caused that the pagebackrefs and index refs were set
%   incorrectly.

\ifpdf
%
% final / draft document options
\usepackage{graphicx}                   % for inserting pdf-graphics.
                                        % options final / draft
\ifdraft{%
% Use biber/biblatex
\usepackage[backend=biber,
            style=ieee,
            sorting=nyt,
            backref=true,
           ]{biblatex}

\usepackage[%pdftex,%
            naturalnames=true,
            linktocpage,
            hyperindex,
            colorlinks,
            urlcolor=dkred,          %\href to external url
            filecolor=dkmagenta,     %\href to local file
            linkcolor=dkred,         %\ref and \pageref
            citecolor=dkgreen,       %\cite
            plainpages=false,
            pdfpagelabels,
            pdfpagemode=UseOutlines, % means use bookmarks (None,UseOutlines)
            % bookmarksopen=false,   % would show the whole hierarchy if true
            bookmarksnumbered=true,
            pdfpagelayout=OneColumn, % SinglePage,OneColumn,TwoColumnLeft,...
            pdfview=FitH, % FitB,FitBH,FitBV,Fit,FitH,FitV
            pdfstartview=FitH, % FitB,FitBH,FitBV,Fit,FitH,FitV
            ]{hyperref}
}{%
% Use biber/biblatex
\usepackage[backend=biber,
            style=ieee,
            sorting=nyt
           ]{biblatex}

\usepackage[%pdftex,%
            naturalnames=true,
            linktocpage,hyperindex,
            colorlinks,
            urlcolor=dkred,          %\href to external url
            filecolor=dkmagenta,     %\href to local file
            linkcolor=dkred,         %\ref and \pageref
            citecolor=dkgreen,       %\cite
            plainpages=false,
            pdfpagelabels,
            pdfpagemode=UseOutlines, % means use bookmarks (None,UseOutlines)
            % bookmarksopen=false,   % open the whole hierarchy if true!
            bookmarksnumbered=true,
            pdfpagelayout=OneColumn, % SinglePage,OneColumn,TwoColumnLeft,...
            pdfview=FitH, % FitB,FitBH,FitBV,Fit,FitH,FitV
            pdfstartview=FitH, % FitB,FitBH,FitBV,Fit,FitH,FitV
            ]{hyperref}
}


%
% Ensure that the links of the images point to the top of the images and not
% to the caption
%
\usepackage[figure]{hypcap}

% %
% % Ensure that pdfLaTeX do the same spacing as LaTeX
% %
\pdfadjustspacing=1 
% %
\else   % i.e. if not pdf

\usepackage[active]{srcltx}             % insert links into the dvi to jump
\usepackage{graphicx}                   % for inserting eps-graphics.
                                        % options final / draft
                                        % into the sources directly.
\ifdraft{%
% Use biber/biblatex
\usepackage[backend=biber,
            style=ieee,
            sorting=nyt,
            backref=true,
           ]{biblatex}

\usepackage[ps2pdf,%
            % plainpages=false,
            linktocpage,
            hyperindex,
            % pdfpagelabels,
            pdfpagemode=UseOutlines,
            pdfstartview=FitH]{hyperref}
}{%
% Use biber/biblatex
\usepackage[backend=biber,
            style=ieee,
            sorting=nyt
           ]{biblatex}
\usepackage[ps2pdf,%
            % plainpages=false,
            linktocpage,
            hyperindex,
            % pdfpagelabels,
            pdfpagemode=UseOutlines,
            pdfstartview=FitH]{hyperref}
}

%\usepackage[ps2pdf]{hyperref}

\fi  % end of if pdf or not

% --------------------------------------------------------------------------

% Allow the use of international characters
\AtBeginDocument{%
  \hypersetup{%
             pdftitle={\thesisTitle},%
             pdfsubject={Tesis de Licenciatura},%
             pdfauthor={\thesisAuthor},%
             pdfkeywords={\thesisKeywords}
            }%
}


%\usepackage{sty/algorithmic}            % algorithmic environment


\usepackage{rotating}                   % allow block rotation


%%%%%%%%%%%%%%%%%%%%%%%%%%%%%%%%%%%%%%%%%%%%%%%%%%%%%%%%%%%%%%%%%%%%%%%%%%%%%%%

%\sloppy

%
% Some own font definitions
%
\DeclareMathAlphabet{\mathpzc}{OT1}{pzc}{m}{it}
\DeclareMathAlphabet{\mathpss}{OT1}{cmss}{m}{sl}

%
% page layout
%

\usepackage{vmargin}
\setpapersize{USletter}

% For letter-paper printing
\setmarginsrb{33mm}{8mm}{23mm}{7mm}{15pt}{15pt}{7mm}{12mm}
%\setlength{\headheight}{15pt}         % fancy headers wanted this

%
% Fraction of Float Object / Text
%

\renewcommand{\topfraction}{0.95}       % how much of top of page should be 
                                        % allowed to be float object?
\renewcommand{\bottomfraction}{0.95}    % how much of bottom of page should be
                                        % allowed to be float object?
\renewcommand{\textfraction}{0.05}      % how much of page must be text?

\usepackage{fancyhdr}                   % fancy page headers

\usepackage{lastpage}

%
% header and footer layout (needs package fancyhdr)
%
\newcommand{\copyrightfooter}{\tiny{\copyright \the\year ---
    \thesisAuthorShort \qquad Uso exclusivo ITCR}}
%
\newcommand{\draftfoot}%
  {\ifdraft{\textcolor{dkblue}{\tiny\textsl{Borrador: \today}}{}}
           {}
}

\pagestyle{fancy}
\renewcommand{\chaptermark}[1]{\markboth{{\small
    \thechapter\hspace*{1mm}#1}}{}}
\renewcommand{\sectionmark}[1]{\markright{{\small
    \thesection\hspace*{1mm}#1}}{}}
%\lhead[{\small\textsc\Roman{\thepage}}]{\fancyplain{}%
\lhead[{\small\thepage}]{\fancyplain{}%
        {{\slshape \small\nouppercase{\leftmark}}}}
\chead[]{}
\rhead[\fancyplain{}%
%        {{\slshape \small\nouppercase{\rightmark}}}]{{\small\textsc\Roman{\thepage}}}
        {{\slshape \small\nouppercase{\rightmark}}}]{{\small\thepage}}
\lfoot[]{\draftfoot}
\ifbook{%
  \cfoot[]{}
}{
  \cfoot[\copyrightfooter]{\copyrightfooter}
}
\rfoot[\draftfoot]{}
\renewcommand{\headrulewidth}{0.5pt}
\renewcommand{\footrulewidth}{0pt}

%
% Caption style for tables
% Requires the packages caption2 and ftcap
% (caption2 required this, but is obsolete now:)
%
%\newcaptionstyle{tablecaptionstyle}{%
%  \renewcommand\captionlabelfont{\normalsize\bf}
%  \renewcommand\captionfont{\normalsize}
%  \usecaptionstyle{hang}%
%}

% For caption v3:
\captionsetup[table]{position=top,format=hang,textfont={normalsize},labelfont={normalsize,bf}}

\newcommand{\tablecaption}[2][foo]{%
  \ifthenelse{\equal{#1}{foo}}{%
    %\captionstyle{tablecaptionstyle}%
    \caption{#2}%
  }
  {%
    %\captionstyle{tablecaptionstyle}%
    \caption[#1]{#2}%
  }
}
\addto\extrasspanish{\renewcommand{\tablename}{Tabla}}
\addto\extrasspanish{\renewcommand{\listtablename}{\'Indice de tablas}}

%
% paragraph layout
%
\renewcommand{\baselinestretch}{1.1}    % line spacing
\parindent0em                           % indentation width of first line
\parskip1.3ex                           % space between paragraphs

%
% document consists of
% chapter - section - subsection - subsubsection - paragraph - subparagraph
%
\setcounter{secnumdepth}{2}             % depth of section numbering
\setcounter{tocdepth}{2}                % depth of table of contents

% For biblatex
\addbibresource{literatura.bib}

%
% prepares index from entries like \index{word} or \index{group!word}.
% don't forget to call "makeindex filename" for final index generation.
%
\makeindex                            %% for package makeidx.sty
%\newindex{default}{idx}{ind}{Index}  %% for package index.sty

\newcommand{\octave}{GNU/Octave}


%
% prepares notation or nomenclature 
%
%\makeglossary
\makenomenclature

%%% Local Variables: 
%%% mode: latex
%%% TeX-master: "main"
%%% End: 
