\chapter{Conclusions}
\label{ch:conclusions}

The experiments performed so far have shown a tendency of the variational autoencoder architectures to blur the reconstructed images. On the other side the AnoGAN has more promising results with better reconstruction images.

A modification to the AnoGAN architecture is proposed, allowing to considerably improve the reconstruction time.

As future work, the objective must be to improve the generator of the GAN. One possible strategy is in the implementation of a disentangling representation of the generator latent space \cite{Higgins2016}. Another possible approach is the use of a regularized in the latent space of the generator, in a similar way as proposed in \cite{Karras2018}. In order to validate the possible improvement in the generator, metrics that evaluate the GAN performance needs to be explored, along with metrics for the evaluation of the anomalies.

Finally, a segmentation process should be implemented in order to mark possible anomalous regions.
