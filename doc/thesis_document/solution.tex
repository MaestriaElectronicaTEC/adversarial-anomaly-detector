\chapter{Proposed solution}
\label{ch:solution}

\section{Data preprocessing}

The data acquired for this project used a camera of an iPhone 8 that captures in a resolution of 1920$\times$1080. The tomato data was captured in tomato crops located in the Alajuela province of Costa Rica (10$^{\circ}$01'39.2"N 84$^{\circ}$18'31.7"W). In the appendix \ref{ch:appendix} is described the capture protocole.

The comparison of anomaly detection models uses 191520 images with a size of 50$\times$50 each. For this set of images, 80\% was used for testing and 20\% used for validation. This training and validation data contains only healthy tomato plants. A set of 798 images is used for testing, and in this case, it also depicts unhealthy tomato plants.

Before using these images for training the models, a data normalization was applied to linearly map the image values to the range between -1 to 1. Additionally, image data augmentation as rotation, zoom, width shift, among other image transformations were implemented.

\section{Architecture experiment}

In order to evaluate the selected architectures for anomaly detection in tomato plants, an experiment of three architectures are proposed: one to explore the adversarial anomaly detector architecture, another to evaluate the spatial variational autoencoder and a third experiment to make use of the gaussian-mixture variational autoencoder.

All the architectures use the same dataset and for each one its reconstruction of an input image is tested by measuring the distance between the original and reconstructed images using the mean square error (MSE). The expected behavior here is that, as the model is only trained with healthy images of tomato, if some input image presents a possible disease, the model should not be able to reconstruct the affected area, and that reconstruction error or dissimilarity is an indication of the presence of an anomaly.

\section{Contribution}

The architecture proposed in this project is an adversarial anomaly detector inspired in the AnoGAN. This architecture is composed of the typical generator and discriminator of a GAN, but with the addition of an encoder \begin{math}E(x) = z\end{math} that is connected with the generator input.

The training process of this approach has two stages: First the the GAN is trained with the conventional methods, but exclusivelly using healthy images of tomato, where the generator receives random noise as input for the latent variables. The second stage is training the encoder that has to learn how to map the query input image into the latent space.

During training of the encoder, the generator and discriminator models remain unchanged. Also, the hidden layers of the discriminator are used to extract features of the synthetic images and the input images. The goal is to train the encoder in a way that it forces the generator to create images with similar features as the ones of the query image.

The proposed loss function for this anomaly detector model is the following:

\begin{equation}
 \mathcal{L}\left(\mathbf{E}(x)\right)=(1-\lambda) \cdot \mathcal{L}_{R}\left(\mathbf{E}(x)\right)+\lambda \cdot \mathcal{L}_{D}\left(\mathbf{E}(x)\right)
\end{equation}

where \begin{math}\lambda\end{math} weights both terms of the function and $\mathbf{E}$ is the Encoder. \begin{math}\mathcal{L}_{R}\end{math} (reconstruction error) and \begin{math}\mathcal{L}_{D}\end{math} (discriminator error) are defined as:

\begin{equation}
 \mathcal{L}_{R}\left(\mathbf{E}(x)\right)=\left\|\mathbf{x}-G\left(\mathbf{E}(x)\right)\right\|
\end{equation}

\begin{equation}
 \mathcal{L}_{D}\left(\mathbf{E}(x)\right)=\left\|\mathbf{f}(\mathbf{x})-\mathbf{f}\left(G\left(\mathbf{E}(x)\right)\right)\right\|
\end{equation}

where \begin{math}\mathbf{f}(x)\end{math} represents the feature extractor of the discriminator.

This approach overcomes one of the disadvantages of the AnoGAN that is its long computational time. This modification of the AnoGAN architecture is similar to the one presented in \cite{Schlegl2019}.
